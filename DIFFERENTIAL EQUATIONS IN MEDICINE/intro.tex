\chapter*{\textbf{INTRODUCTION}}
\thispagestyle{empty}
\numberwithin{equation}{chapter}



\par~~~~~~~~~~ Mathematics is the science that deals with the logic of shape,quantity and arrangement and the applications of mathematics involves the study of physical, biological or sociological world. Prior to taking part in this mathematics project we have been discussing the topic "\textbf{Applications of differential equation in medicine}".

\par ~~~~~~~~~~~~~~Differential Equations can be used to explain and predict new facts about everything that change continuously. For example, the orbit of a spaceship, amount of money in a saving bank, amount of deformation of elastic structures, description of radio waves, size of biological population and on and on.  Differential equations originates whenever some universal law of nature is expressed in terms of a mathematical variable and at least one of its derivatives. Interestingly, the date of birth of differential equation is taken to be November,11,1675, when \textit{Gottfried Wilhelm Leibniz} employed integral calculus for the first time to find the area under the graph of a function  $y=f(x)$  by introducing both the symbols $\int$ and dy.
 
\par ~~~~~~~~~~~~Differential equations are a mathematical means to describe the natural world. Several laws defining the behaviour of the natural world are relations involving rates at which things occur. To express the laws mathematically these relations become equations and the rates become derivatives. The process by which scientists and engineers use differential equations to understand physical phenomena can be broken down into three steps; Data collection, modelling process and the last one is to solve mathematically the ideal problem and compare the solutions with the measurements of the real phenomena.

\par ~~~~~~~~~~~~~~~~The history of mathematics in the biomedical sciences can be traced back at least to 1798, when Thomas Malthus published his famous growth law of the human population. The ‘logistic’ growth rate proposed by \textit{Verhulst} carried over to other models in population dynamics. The medical applications of mathematics belong to a large number of branches. Rather than zooming in on a specific one, we have selected a sample of distinct contributions in oncology, cardiology, epidemiology and pathology to show the possibilities of the mathematical approach and how it can enrich topics of different nature.

\par ~~~~~~~~~~The dynamics of tumor growth expresses the dependence of the tumor site on time, while the problem in epidemics deals with the occurrence, spreading and control of contagious disease.  Mathematical model in pharmacokinetics often describe the evolution of pharmacological process in terms of system of linear or non linear ODEs. Diabetes detection is the study on maintaining a level of glucose concentration in human body, and the chapter on cardiology is to predict the heart rate and blood pressure dynamics under exercise stress.

\par ~~~~~~~~~~Finally, let us point out that the application of mathematics to medicine is going through a time of great scientific interest. In turn, it is also true that mathematics owes much of its inspiration and vigorous development to the natural sciences and, increasingly, also to biology, psychology, economy, social sciences and medicine.
 
\par ~~~~~~~~This short paper will provide an overview of the ODE modelling framework, and present examples of how ODEs can be used to address problems in medicine.


